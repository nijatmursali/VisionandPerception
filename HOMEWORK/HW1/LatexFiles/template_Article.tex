\documentclass[]{article}

%opening
\title{Vision and Perception}
\author{Nijat Mursali}
\usepackage{tikz}
\usetikzlibrary{calc,intersections,through,backgrounds}
\begin{document}

\maketitle

\begin{abstract}
	This document is used to illustrate all the homework exercises that has been given during Vision and Perception course in 2020/21. There are overall 10 exercises that needed to be done during this time. 


\end{abstract}

\section{Homework 1 - Degenerate Conic}
As mentioned during the video of Geometric Parameters, we needed to compute the $M_{2}$ as we did for $M_{1}$ and then computing the null space of $M_{1}$ and verifying that is 2, then computing the cross product of null space vectors and after normalizing to obtain the $x_{3} = 1$, we needed to think about the result.
\vspace{0.4em}

The idea is here space of $M_{1}$ and verifying that is 2, then computing the cross product of null space vectors and after normalizing to obtain the $x_{3} = 1$, we needed to think about the result.	

\begin{tikzpicture}
\draw (-1.5, 0) coordinate(A) -- (2,0.2) coordinate (B) node [black, scale=1] {$l_{1}$};
\draw (-1,-1) coordinate(C) -- (1.5,1.5) coordinate (D) node [black, scale=1] {$l_{2}$};
\node[red,scale=3] at (intersection of  A--B and C--D){.};
\end{tikzpicture}

Thus, we need to consider the two intersecting lines that we have seen in our lecture: 

$l_{1} = (0.2500, 3.200, 1.0000)^T$  

$l_{2} = (-2.0000, 0.5000, 1.0000)^T$ 

$x = l_{1}xl_{2} = (-0.4138, 0.3448, 1.0000)^T$
\vspace{0.2em}

$M_{1} = l_{1}l_{1}^T = 
\left( {\begin{array}{*{20}c}
	0.0625 & 0.8000 & 0.2500 \\
	0.8000 & 10.2400 & 3.2000 \\
	0.2500 &  3.2000 & 1.0000   
	\end{array} } \right)
$

Thus, by computing the degenerate conic for $l_{2}$, we could get something following:

$M_{2} = l_{2}l_{2}^T = 
\left( {\begin{array}{*{20}c}
	4.0000 & -1.0000 & -2.0000 \\
	-1.0000 & 0.2500 & 0.5000 \\
	-2.0000 &  0.5000 & 1.0000   
	\end{array} } \right)
$

As we see from the book, the null space can be computed by span as following:

\centerline {$Null(M_{1}) = Span\{\left( {\begin{array}{*{20}c}
	0.2210 \\
	0.2751 \\
	-0.9357  
	\end{array} } \right),\left( {\begin{array}{*{20}c}
	-0.9724 \\
	 0.1353 \\
	-0.1899  
	\end{array} } \right) \}$ }

and 

\centerline { $Null(M_{2}) = Span\{\left( {\begin{array}{*{20}c}
	0 \\
	0.8944 \\
	-0.4472  
	\end{array} } \right),\left( {\begin{array}{*{20}c}
	-0.4880 \\
	-0.3904 \\
	-0.7807  
	\end{array} } \right) \}$ }

Finally, we need to calculate the cross-product of the null-space components of $M_{1}$ as:

$M_{1} = (0.2210, 0.2751, -0.9357)x(-0.9724, 0.1353, -0.18,99) = (0.0744, 0.9518, 0.2974)^T$
\vspace{0.2em}
then, for $M_{2}$ it will be like:
\vspace{0.2em}

$M_{2} = (0, 0.8944 - 0.4472) x (-0.4880, -0.3904, -0.7807) = (-.8728, 0.2182, 0.4365)^T$

We come up to the result that these two results are no more than the original lines, $l_{1}$ and $l_{2}$ with unitary norm.  

\section{Homework 2 - Five Points Define a Conic}
For each point the conic passes through 

\centerline {$ax_{i}^2 + bx_{i}y_{i} + cy_{i}^2 + dx_{i} + ey_{i} + f = 0$ }
\vspace{0.4em}

As we see from the equation, we need to determine the parameters of the equation with at least five points. So, for this homework we just chose 4 points in xy plane with different values, thus we have the following set of equations:

\centerline {
	$\left[ {\begin{array}{*{20}c}
			0 & 0 & 16 & 0 & 4 & 1 \\
			1 & 1 & 1  & 1 & 1 & 1 \\
			9 & 24 & 64  & 3 & 8 & 1 \\
			36 & 36 & 36  & 6 & 6 & 1 \\
			49 & 21 & 9  & 7 & 3 & 1\\   
	\end{array} } \right] 
	\left[ {\begin{array}{*{20}c}
			a \\ 
			b \\ 
			c \\ 
			d \\ 
			e \\ 
			f    
	\end{array} } \right] = 0 $
}

from which we obtain the following values:

\centerline {
	$a = 0.0876$, $b = 0.0258$, $c = 0.0532$, $d = -0.7038$, $e = -0.4629$, $f = 1$,  
}

thus giving us following result:



\section{Homework 3 - Compute the DLT algorithm}
In order to solve this problem, we needed to take one picture and apply the algorithm to check the coordinates of the shape. The idea is to pick 4 points on two different images by hand and apply the DLT algorithm showing the calculations. Additionally, the point selection is arbitrary. 

Thus, we consider the following picture:

The fundamental goal of ours is to apply homography so that the outline shape of object forms a rectangle. The homogeneous coordinates for the shape are:

\centerline {
	$A = $
}

and our target shape has the coordinates: 

\centerline {
	$B = $
}

thus yielding the following equation: 

\centerline {
	$B = AH$
}

in which we ought to find the homography that produces such transformation. The matrix A is non-square, thus it cannot be inverted, we have to resort to the Direct Linear Transformation algorithm, in which at least three non-co-linear points are used to find the 8 independent terms of the transformation matrix. Since our settings have four points, our system is over determined, from which we derive the following transformation matrix:

\centerline {
	$H = $
}

from which we finally obtain the following transformation:



\section{Homework 4 - Affine Transformations}
The task for this exercise is to show that an affine transformation preserves both parallel lines and area. 
\section{Homework 5 - Homography keeps lines tangent to conics}
Consider the following line $l$, tanget to the conic $C$ at the point $x$, 

From this property, we know that $l = Cx$ and $x^Tl=x^TCx = 0$. Using the homography $H$ on $x$, we get $x'=Hx$, and to reverse this transformation, we can use the inverse homography $H^-1$. With the aformentioned properties, we can rewrite the equations like so:

\centerline {
	$x^TCx = (H^-1x')^TC(H^-1x') = 0$
}
this, in turn, is equivalent to :

\centerline {
	$x^TCx = x'(H^-1)^TCH^-1x' = x'C'x' = 0$.
}

From the relation $l = Cx$, we have:

\centerline {
	$(H^-1)^Tl = (H^-1)^TCx$,
}
which can be rewritten to:
\centerline {	
	$(H^-1)^Tl = (H^-1)^TC(H^-1H)x$.
}

As we have previously seen: $C' = (H^-1)^TCH^-1$, and so the previous equation becomes:

\centerline {
	$(H^-1)^Tl = C'Hx$
}

thus finally arriving at:

\centerline {
	$l' = C'x'$
}

which implies that $l'$ is also tangent to $C'$.

\section{Homework 6 - SVD for given matrix}
As we have seen from the example, the matrix A which we out to decompose is the following: 

\centerline{ $A = \left[ {\begin{array}{*{20}c}
		2 & 3 & 1 \\
		1 & 4 & -2   
		\end{array} } \right]$ }
	
The singular value decomposition (SVD) property proposes that any given matrix can be decomposed in three matrices that is:

\centerline { $A = U \Sigma V^T$}

where $\Sigma$ is a rectangular diagonal matrix and, $U$ and $V^T$ are orthogonal matrices. To obtain these matrices, we first compute the product $A^TA$ as following:

\centerline {
	$A^TA = (U \Sigma V^T)^T U \Sigma V^T = V (\Sigma^T \Sigma)V^T $ 
}
which will be equal to 

\centerline{ $A^TA = \left[ {\begin{array}{*{20}c}
		5 & 10 & 0 \\
		10 & 25 & -5 \\ 
		0 & -5 & 5   
		\end{array} } \right]$ }
	
from this relation, we extract the eigenvalues of $A^TA$ as follows:
	
\centerline {
	$(A^TA - I \lambda) x = 0$
}

in which, in order to obtain a non-trivial solution, the following property must hold:

\centerline {
	$det(A^TA - I \lambda) = 0$
}

which gives the following equation:

\centerline {
	$(5- \lambda) (25 - \lambda)(5 - \lambda) - 25(5 - \lambda) - 100(5 - \lambda) = 0 $
}

From this equation, we arrive at the following solutions: 

\centerline {
	$\lambda_{1} = 30$, $\lambda_{2} = 5$, $\lambda_{3} = 0$ 
}

To obtain the eingenvectors, we must find the vectors that lie in the null-space of the resulting matrices once
each eigenvalue is substituted:

For $\lambda = 30$: 

\centerline{ $\left[ {\begin{array}{*{20}c}
		-25 & 10 & 0 \\
		10 & -5 & -5 \\ 
		0 & -5 & -25   
		\end{array} } \right] x = 0$,  }

Giving us the following expressions: 

\centerline {
	$-25x_{1} + 10x_{2} = 0$, 
}
\centerline {
	$10x_{1} - 5x_{2} - 5x_{3} =0$, 
}
\centerline {
	$-5x_{2} - 25x_{3} = 0$	
}

from this linear set of equations, we arrive at the following solution: 

\centerline {
	$x = [-1, -2.5, 0.5]^T$,
}

which is then normalized: 

\centerline {
	$x = [-0.3651, -0.9129, 0.1826]^T$.
}

Repeating the same procedure for the other eigenvalues, we arrive at: 

\centerline {
	$x = [-0.4472, 0, -0.8944]^T$ and $x = [0.8165, -0.4082, -0.4082]^T$
}

Like so, we obtain the matrix $V$, compose of the column vectors: 

\centerline{ $V = \left[ {\begin{array}{*{20}c}
		-0.3651 & -0.4472 & 0.8165 \\
		-0.9129 & 0 & -0.4082 \\ 
		0.1826 & -0.8944 & -0.4082   
		\end{array} } \right]$ 
}
	
furthemore, we also obtain the matrix $\lambda$, which is the square root of the diagonal matrix composed of the calculated eigenvalues: 

\centerline{ $\Sigma = \left[ {\begin{array}{*{20}c}
		\sqrt{30} & 0 & 0 \\
		0 & \sqrt{5} & 0 \\ 
		0 & 0 & 0   
		\end{array} } \right]$ 
}

since the last row is a product of any above row, we can write:

\centerline{ $\Sigma = \left[ {\begin{array}{*{20}c}
		\sqrt{30} & 0 & 0 \\
		0 & \sqrt{5} & 0   
		\end{array} } \right]$ 
}

The last step is to determine the orthonormal matrix $U$. This can be obtained by the following product:

\centerline {
	$AV = U \Sigma V^TV$
} 

where we cross out the $\Sigma V^TV$ that gives us only $AV = U$. Hence:

\centerline{ $\left[ {\begin{array}{*{20}c}
		2 & 3 & 1 \\
		1 & 4 & -2   
		\end{array} } \right] 
			\left[ {\begin{array}{*{20}c}
		-0.3651 & -0.4472 & 0.8165 \\
		-0.9129 & 0 & -0.4082 \\ 
		0.1826 & -0.8944 & -0.4082  
		\end{array} } \right] = U \left[ {\begin{array}{*{20}c}
		\sqrt{30} & 0 & 0 \\
		0 & \sqrt{5} & 0   
		\end{array} } \right]$ 
}

which gives us the following matrix $U$:

\centerline {
	$U = \left[ {\begin{array}{*{20}c}
		-0.6 & -0.8 \\
		-0.8 & 0.6   
		\end{array} } \right]$
}

\section{Homework 7 - Projective Transformation}  
The task here is to find the projective transformation $H$ and define the type of quadric from the quadric equation.

Firstly, let's clarify some points we have learned in our lectures. 

\centerline {
	3D point in $R^3$ which is $X = (X, Y, Z)^T$ in $P^3 : (x_1, x_2, x_3, x_4)^T$,
}

\centerline {
	Euclidean frame $\pi : ax + by + cz +d = 0$, where
	$\pi^TX = 0 => (a b c d) ^T\left[ {\begin{array}{*{20}c}
		x_1 \\
		x_2 \\ 
		x_3 \\
		x_4  
		\end{array} } \right] = 0$	
}

Thus, 

\centerline {
	$X = (X_1, X_2, X_3, X_4)$ derives, 
}
\vspace{0.2em}
\centerline {
		$X^TAX = 4x_1^2 + 4x_1x_2 - 2x_1x_3 + 2x_1x_4 + 5x_2^2 - 2x_2x_4 + 2x_3^2 + 2x_3x_4 + 2x_4^2 = 0$
}

Then, we get the following A matrix,

\centerline {
	$A = a_ia_j = \left[ {\begin{array}{*{20}c}
		4 & 2 & -1 & 1 \\
		2 & 5 & 0 & -1 \\ 
		0 & 0 & 2 &  1 \\
		0 & 0 & 0 & 2  
		\end{array} } \right] = 0 $ 
}

Then from the following equation we are getting the $\lambda$ values. 

\centerline {
	$|\lambda I - A| = \left[ {\begin{array}{*{20}c}
		\lambda - 4 & -2 & 1 & -1 \\
		-2 & \lambda - 5 & 0 & 1 \\ 
		1 & 0 & \lambda - 2 &  -1 \\
		-1 & 1 & -1 & \lambda - 2  
		\end{array} } \right] = 0$
}

\centerline {
	$\lambda ^4 - 13 \lambda ^3 + 52 \lambda ^2 - 81 \lambda + 25 = 0$, 
}

\centerline {
	$\lambda _1 = 6.6637$, $\lambda _2 = 3.6360$, $\lambda _3 = 2.7153$, $\lambda _4 = -0.0151$
}

Then, from the matrix we get the $V$ values, 

\centerline {
	$V_1 = \left[ {\begin{array}{*{20}c}
		-9.2387 \\
		-11.7071 \\ 
		2.1953 \\
		1  
		\end{array} } \right]$, 
	$V_2 = \left[ {\begin{array}{*{20}c}
		0.9476 \\
		-0.6564 \\ 
		0.0319 \\
		1  
		\end{array} } \right]$, 
	$V_3 = \left[ {\begin{array}{*{20}c}
		-0.5125 \\
		0.9964 \\ 
		2.1143 \\
		1  
		\end{array} } \right]$,
	$V_4 = \left[ {\begin{array}{*{20}c}
		-0.6963 \\
		0.4770 \\ 
		-0.8417 \\
		1  
		\end{array} } \right]$,  
}

Then, from the formula of $U_i = AV_i/\sigma _i$, we tried to find the $U$ value and got the following values:

\section{Homework 8 - Point Equations}
The task here is to define the pairs of point equations for the direction and pair of plane equations for coordinate planes. Thus, for a given homogeneous coordinate system in $P^3$, we have the following set of point equations that define the axes:

As we have learned from the lecture 

\centerline {
	$ax_1 + bx_2 + cx_3 + dx_4 = 0$ gives $(a: b: c: d)$
}

where 

\centerline {
	$\pi ^TX = 0$ gives  $(a b c d)^T \left[ {\begin{array}{*{20}c}
		x_1 \\
		x_2 \\ 
		x_3 \\
		x_4   
		\end{array} } \right] = 0$
}

\centerline {
	$x = (1, 0, 0, 1)^T$, $y = (0, 1, 0, 1)^T$ and $z = (0, 0, 1, 1)^T$
} 

and the coordinate planes:

\centerline {
	$x - y = (0, 0, 1, 1)^T$, $y - z = (1, 0, 0, 1)^T$ and $x - z = (0, 1, 0, 1)^T$
}

Given the equation of the quadric used in the previous exercise:

\centerline {
	$X^TAX = 4x_1^2 + 4x_1x_2 - 2x_1x_3 + 2x_1x_4 + 5x_2^2 - 2x_2x_4 + 2x_3^2 + 2x_3x_4 + 2x_4^2$
}

which can be rewritten as: 

\centerline {
	$Q = \left[ {\begin{array}{*{20}c}
		4 & 2 & -1 & 1\\
		2 & 5 & 0 & -1 \\ 
		-1 & 0 & 2 & 1 \\
		1 & -1 & 1 & 2  
		\end{array} } \right]$,
}

as well as the following coordinates of a plane: 

\centerline {
	$\pi = (2, 3, 1, 1)^T$
}

thus giving us the following null space:

\centerline {
	$M_ \pi = \left[ {\begin{array}{*{20}c}
		−0.775 & −0.258 & −0.258 \\
		0.604 & -0.132 & -0.132 \\ 
		-0.132 & 0.956 & -0.044 \\
		-0.132 & -0.044 & 0.956  
		\end{array} } \right]$,
}

From the relation of  $C = M_ \pi ^TQM_ \pi$, we obtain the relation for the conic:

\centerline {
	$C = \left[ {\begin{array}{*{20}c}
		2.617 & 0.717 & −1.437 \\
		0.717 & 2.742 & 1.358 \\ 
		-1.437 & 1.358 & 1.973  
		\end{array} } \right]$,
}

thus, the conic is represented as following:

\centerline {
	$C = 2.617x_1^2 + 0.358x_1x_2 + 2.742x_2^2 - 0.718x_1x_3 + 0.679x_2x_3 + 1.973x_3^2$
}

\section{Homework 9 - Equation of Conic}



\section{Homework 10 - Proof for $cos(\alpha)$}  

Points on the plane at infinity $(\pi _ \infty)$, which may be written as $X_ \infty = (d^T, 0)^T$ are mapped to the image plane by a general camera $P = CR[I|t]$ as 

\centerline {
	$x = PX_ \infty = CR[I|t](d^T, 0)^T = CRd$
}

Thus, in this case $H = CR$ is the planar homography between $(\pi _ \infty)$ and the image plane. Since the absolute conic $(\Omega _ \infty)$ is on $(\pi _ \infty)$, we can compute its image as 

\centerline {
	$\omega = (CC^T)^-1 = C^-TC^1$
}
 
Like  $(\Omega _ \infty)$, $\omega$ is an imaginary point conic with no real points. It cannot really be observed in an image. $\omega$ is dependent only on the internal parameters of the camera and is independent of the camera's position or orientation. Thus,
it follows from above that the angle between two rays is given by the simple equation 

The above expression is independent of the choice of the projective coordinate on the image. To see this consider any 2D projective transformation $H$. The points $x_i$ are transformed to $Hx_i$, and $\omega$ transforms to $H^-T \omega H^-1$. Hence, the expression for $cos \omega$ is unchanged. THus it will still be valid for any projective frame. 


\section{Homework 11 - Euclidean Rotation}

The idea here is that dual conic here the duality is between planes and points. Thus, $Q_ \infty ^*$ is made by planes tangent to $(\Omega _ \infty)$, therefore any plane that belongs to $Q_ \infty ^*$ envelope is tangent to $\Omega$. 

\centerline {
	$\pi = \Omega _ \infty X = \pi ^TQ_ \infty ^* = 0$ 
}

\centerline {
	$Q_ \infty ^* = \left[ {\begin{array}{*{20}c}
		I & 0 \\
		0^T & 0   
		\end{array} } \right]$,
}

When we consider the Euclidean transformation represented by the matrix 

\centerline {
	$H_E = \left[ {\begin{array}{*{20}c}
		R & 0 \\
		0^T & 1   
		\end{array} } \right] =  \left[ {\begin{array}{*{20}c}
		cos \theta & -sin \theta & 0 & 0 \\
		sin \theta & cos \theta  & 0 & 0 \\ 
		0 & 0 & 1 & 0 \\ 
		0 & 0 & 0 & 1   
		\end{array} } \right] $,
}

This is a rotation by $ \theta $ about the z-axis with a zero translation. Geometrically it is evident that the family of XY -planes orthogonal to the rotation axis are simply rotated about the Z -axis by this transformation.

This means that there is a pencil of fixed planes orthogonal to the Z -axis. The planes
are fixed as sets, but not point-wise as any (finite) point (not on the axis) is rotated in horizontal circles by this Euclidean action. Algebraically, the fixed planes of H are the eigenvectors of $H^T$.

Corresponding eigenvectors of $H_E^T$ are 

\centerline {
	$E_1 = \left( {\begin{array}{*{20}c}
		1 \\
		i \\ 
		0 \\
		0   
		\end{array} } \right)$, $E_2 = \left( {\begin{array}{*{20}c}
		1 \\
		-i \\ 
		0 \\
		0   
		\end{array} } \right)$, 	$E_3 = \left( {\begin{array}{*{20}c}
		0 \\
		0 \\ 
		1 \\
		0   
		\end{array} } \right)$, 	$E_4 = \left( {\begin{array}{*{20}c}
		0 \\
		0 \\ 
		0 \\
		1   
		\end{array} } \right)$
}

The eigenvectors $E_3$ and $E_4$ are degenerate. Thus there is a pencil of fixed planes which is spanned by these eigenvectors. The axis of this pencil is the line of intersection of the the planes (perpendicular to the Z -axis) with $\pi _ \infty$, and the pencil includes $\pi _ \infty$ .

The example also illustrates the connection between the geometry of the projective
plane, $P_2$ , and projective 3-space, $P_3$ . A plane $\pi$ intersects $\pi _ \infty$ in a line which is the line at infinity,  $l_ \infty$ , of the plane $\pi$. A projective transformation of IP 3 induces a subordinate plane projective transformation on $\pi$.

For defining planes, we have to find the eigenvectors of $H_E^T$ by:

\centerline {
	$[H_E^T - \lambda I]v = 0 $ which gives $\left[ {\begin{array}{*{20}c}
		cos \theta - \lambda & sin \theta & 0 & 0 \\
		-sin \theta & cos \theta - \lambda & 0 & 0 \\ 
		0 & 0 & 1 - \lambda & 0 \\ 
		0 & 0 & 0 & 1 - \lambda   
		\end{array} } \right] $
}

Thus, we have to find such $\lambda$ that satisfies $det(H_E^T - \lambda I) = 0$

\section{Homework 12 - }

The camera centre $C$ is the point for which $PC = 0$. Numerically, this right null-vector may be obtained from the SVD of $R$. Algebraically, the centre $C = (X, Y,Z,T)^T$, where 

\centerline {
	$X=det([p_2, p_3, p_4])$, $Y=-det([p_1, p_3, p_4])$, 
}

\centerline {
	$Z=det([p_1, p_2, p_4])$, $T=-det([p_1, p_2, p_3])$
}

and the $P$ is 

\centerline {
	$P=[M | -MC] = K[R | -RC ]$
}

We can easily find K nad R by decomposing M as $M = KR$ using the $RQ$ decomposition. The matrix $R$ gives the orientation of camera, whereas $K$ is calibration matrix. 

\centerline {
		$K = \left[ {\begin{array}{*{20}c}
		\alpha _x & s & x_0\\
		0 & \alpha _y & y_0 \\ 
		0 & 0 & 1   
		\end{array} } \right]$ and $R|t = \left[ {\begin{array}{*{20}c}
		r_{11} & r_{12} & r_{13} & t_1 \\
		r_{21} & r_{22} & r_{23} & t_2 \\ 
		r_{31} & r_{32} & r_{33} & t_3    
		\end{array} } \right]$
}

For our problem we have $P$ as:

\centerline {
	$P = $
}

Thus, when we decompose the matrix $M$ as $KR$ using QR-decomposition 

\centerline {
	$P = [M| -MC]$ the centre $C = -R^Tt$
}

For our problem, $M$ and $MC$ equals 

\centerline {
	$M = $, $MC = $
}

Then, we can compute $C$ by

\centerline {
	$C = $
}

We get  $C = $

b) For this exercise, we needed to find the translation vector which can be calculated as 

\centerline {
	$t = -RC$, thus $t = $
}


\section{Homework 13 - }

\section{Homework 14 - }
 
\end{document}
